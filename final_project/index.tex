% Options for packages loaded elsewhere
\PassOptionsToPackage{unicode}{hyperref}
\PassOptionsToPackage{hyphens}{url}
\PassOptionsToPackage{dvipsnames,svgnames,x11names}{xcolor}
%
\documentclass[
  letterpaper,
  DIV=11,
  numbers=noendperiod]{scrartcl}

\usepackage{amsmath,amssymb}
\usepackage{lmodern}
\usepackage{iftex}
\ifPDFTeX
  \usepackage[T1]{fontenc}
  \usepackage[utf8]{inputenc}
  \usepackage{textcomp} % provide euro and other symbols
\else % if luatex or xetex
  \usepackage{unicode-math}
  \defaultfontfeatures{Scale=MatchLowercase}
  \defaultfontfeatures[\rmfamily]{Ligatures=TeX,Scale=1}
\fi
% Use upquote if available, for straight quotes in verbatim environments
\IfFileExists{upquote.sty}{\usepackage{upquote}}{}
\IfFileExists{microtype.sty}{% use microtype if available
  \usepackage[]{microtype}
  \UseMicrotypeSet[protrusion]{basicmath} % disable protrusion for tt fonts
}{}
\makeatletter
\@ifundefined{KOMAClassName}{% if non-KOMA class
  \IfFileExists{parskip.sty}{%
    \usepackage{parskip}
  }{% else
    \setlength{\parindent}{0pt}
    \setlength{\parskip}{6pt plus 2pt minus 1pt}}
}{% if KOMA class
  \KOMAoptions{parskip=half}}
\makeatother
\usepackage{xcolor}
\setlength{\emergencystretch}{3em} % prevent overfull lines
\setcounter{secnumdepth}{5}
% Make \paragraph and \subparagraph free-standing
\ifx\paragraph\undefined\else
  \let\oldparagraph\paragraph
  \renewcommand{\paragraph}[1]{\oldparagraph{#1}\mbox{}}
\fi
\ifx\subparagraph\undefined\else
  \let\oldsubparagraph\subparagraph
  \renewcommand{\subparagraph}[1]{\oldsubparagraph{#1}\mbox{}}
\fi

\usepackage{color}
\usepackage{fancyvrb}
\newcommand{\VerbBar}{|}
\newcommand{\VERB}{\Verb[commandchars=\\\{\}]}
\DefineVerbatimEnvironment{Highlighting}{Verbatim}{commandchars=\\\{\}}
% Add ',fontsize=\small' for more characters per line
\usepackage{framed}
\definecolor{shadecolor}{RGB}{241,243,245}
\newenvironment{Shaded}{\begin{snugshade}}{\end{snugshade}}
\newcommand{\AlertTok}[1]{\textcolor[rgb]{0.68,0.00,0.00}{#1}}
\newcommand{\AnnotationTok}[1]{\textcolor[rgb]{0.37,0.37,0.37}{#1}}
\newcommand{\AttributeTok}[1]{\textcolor[rgb]{0.40,0.45,0.13}{#1}}
\newcommand{\BaseNTok}[1]{\textcolor[rgb]{0.68,0.00,0.00}{#1}}
\newcommand{\BuiltInTok}[1]{\textcolor[rgb]{0.00,0.23,0.31}{#1}}
\newcommand{\CharTok}[1]{\textcolor[rgb]{0.13,0.47,0.30}{#1}}
\newcommand{\CommentTok}[1]{\textcolor[rgb]{0.37,0.37,0.37}{#1}}
\newcommand{\CommentVarTok}[1]{\textcolor[rgb]{0.37,0.37,0.37}{\textit{#1}}}
\newcommand{\ConstantTok}[1]{\textcolor[rgb]{0.56,0.35,0.01}{#1}}
\newcommand{\ControlFlowTok}[1]{\textcolor[rgb]{0.00,0.23,0.31}{#1}}
\newcommand{\DataTypeTok}[1]{\textcolor[rgb]{0.68,0.00,0.00}{#1}}
\newcommand{\DecValTok}[1]{\textcolor[rgb]{0.68,0.00,0.00}{#1}}
\newcommand{\DocumentationTok}[1]{\textcolor[rgb]{0.37,0.37,0.37}{\textit{#1}}}
\newcommand{\ErrorTok}[1]{\textcolor[rgb]{0.68,0.00,0.00}{#1}}
\newcommand{\ExtensionTok}[1]{\textcolor[rgb]{0.00,0.23,0.31}{#1}}
\newcommand{\FloatTok}[1]{\textcolor[rgb]{0.68,0.00,0.00}{#1}}
\newcommand{\FunctionTok}[1]{\textcolor[rgb]{0.28,0.35,0.67}{#1}}
\newcommand{\ImportTok}[1]{\textcolor[rgb]{0.00,0.46,0.62}{#1}}
\newcommand{\InformationTok}[1]{\textcolor[rgb]{0.37,0.37,0.37}{#1}}
\newcommand{\KeywordTok}[1]{\textcolor[rgb]{0.00,0.23,0.31}{#1}}
\newcommand{\NormalTok}[1]{\textcolor[rgb]{0.00,0.23,0.31}{#1}}
\newcommand{\OperatorTok}[1]{\textcolor[rgb]{0.37,0.37,0.37}{#1}}
\newcommand{\OtherTok}[1]{\textcolor[rgb]{0.00,0.23,0.31}{#1}}
\newcommand{\PreprocessorTok}[1]{\textcolor[rgb]{0.68,0.00,0.00}{#1}}
\newcommand{\RegionMarkerTok}[1]{\textcolor[rgb]{0.00,0.23,0.31}{#1}}
\newcommand{\SpecialCharTok}[1]{\textcolor[rgb]{0.37,0.37,0.37}{#1}}
\newcommand{\SpecialStringTok}[1]{\textcolor[rgb]{0.13,0.47,0.30}{#1}}
\newcommand{\StringTok}[1]{\textcolor[rgb]{0.13,0.47,0.30}{#1}}
\newcommand{\VariableTok}[1]{\textcolor[rgb]{0.07,0.07,0.07}{#1}}
\newcommand{\VerbatimStringTok}[1]{\textcolor[rgb]{0.13,0.47,0.30}{#1}}
\newcommand{\WarningTok}[1]{\textcolor[rgb]{0.37,0.37,0.37}{\textit{#1}}}

\providecommand{\tightlist}{%
  \setlength{\itemsep}{0pt}\setlength{\parskip}{0pt}}\usepackage{longtable,booktabs,array}
\usepackage{calc} % for calculating minipage widths
% Correct order of tables after \paragraph or \subparagraph
\usepackage{etoolbox}
\makeatletter
\patchcmd\longtable{\par}{\if@noskipsec\mbox{}\fi\par}{}{}
\makeatother
% Allow footnotes in longtable head/foot
\IfFileExists{footnotehyper.sty}{\usepackage{footnotehyper}}{\usepackage{footnote}}
\makesavenoteenv{longtable}
\usepackage{graphicx}
\makeatletter
\def\maxwidth{\ifdim\Gin@nat@width>\linewidth\linewidth\else\Gin@nat@width\fi}
\def\maxheight{\ifdim\Gin@nat@height>\textheight\textheight\else\Gin@nat@height\fi}
\makeatother
% Scale images if necessary, so that they will not overflow the page
% margins by default, and it is still possible to overwrite the defaults
% using explicit options in \includegraphics[width, height, ...]{}
\setkeys{Gin}{width=\maxwidth,height=\maxheight,keepaspectratio}
% Set default figure placement to htbp
\makeatletter
\def\fps@figure{htbp}
\makeatother

\KOMAoption{captions}{tableheading}
\makeatletter
\makeatother
\makeatletter
\makeatother
\makeatletter
\@ifpackageloaded{caption}{}{\usepackage{caption}}
\AtBeginDocument{%
\ifdefined\contentsname
  \renewcommand*\contentsname{Table of contents}
\else
  \newcommand\contentsname{Table of contents}
\fi
\ifdefined\listfigurename
  \renewcommand*\listfigurename{List of Figures}
\else
  \newcommand\listfigurename{List of Figures}
\fi
\ifdefined\listtablename
  \renewcommand*\listtablename{List of Tables}
\else
  \newcommand\listtablename{List of Tables}
\fi
\ifdefined\figurename
  \renewcommand*\figurename{Figure}
\else
  \newcommand\figurename{Figure}
\fi
\ifdefined\tablename
  \renewcommand*\tablename{Table}
\else
  \newcommand\tablename{Table}
\fi
}
\@ifpackageloaded{float}{}{\usepackage{float}}
\floatstyle{ruled}
\@ifundefined{c@chapter}{\newfloat{codelisting}{h}{lop}}{\newfloat{codelisting}{h}{lop}[chapter]}
\floatname{codelisting}{Listing}
\newcommand*\listoflistings{\listof{codelisting}{List of Listings}}
\makeatother
\makeatletter
\@ifpackageloaded{caption}{}{\usepackage{caption}}
\@ifpackageloaded{subcaption}{}{\usepackage{subcaption}}
\makeatother
\makeatletter
\@ifpackageloaded{tcolorbox}{}{\usepackage[many]{tcolorbox}}
\makeatother
\makeatletter
\@ifundefined{shadecolor}{\definecolor{shadecolor}{rgb}{.97, .97, .97}}
\makeatother
\makeatletter
\makeatother
\ifLuaTeX
  \usepackage{selnolig}  % disable illegal ligatures
\fi
\IfFileExists{bookmark.sty}{\usepackage{bookmark}}{\usepackage{hyperref}}
\IfFileExists{xurl.sty}{\usepackage{xurl}}{} % add URL line breaks if available
\urlstyle{same} % disable monospaced font for URLs
\hypersetup{
  pdftitle={Final Project},
  pdfauthor={Juliet Nwagwu Ume-Ezeoke},
  colorlinks=true,
  linkcolor={blue},
  filecolor={Maroon},
  citecolor={Blue},
  urlcolor={Blue},
  pdfcreator={LaTeX via pandoc}}

\title{Final Project}
\author{Juliet Nwagwu Ume-Ezeoke}
\date{}

\begin{document}
\maketitle
\ifdefined\Shaded\renewenvironment{Shaded}{\begin{tcolorbox}[frame hidden, sharp corners, borderline west={3pt}{0pt}{shadecolor}, boxrule=0pt, breakable, enhanced, interior hidden]}{\end{tcolorbox}}\fi

\renewcommand*\contentsname{Table of contents}
{
\hypersetup{linkcolor=}
\setcounter{tocdepth}{3}
\tableofcontents
}
\hypertarget{executive-summary}{%
\section{Executive Summary}\label{executive-summary}}

\hypertarget{introduction}{%
\section{Introduction}\label{introduction}}

The world urgently needs to transition to 100\% clean, renewable energy.
This fact is particulary difficult to come to terms with in oil-rich
developing countries such as Nigeria. Here, the mining, sale, and use of
abundant fossil fuels is expected to pave the way to economic
prosperity. However, a cursory glance at the country's history reveals
that fossil fuels have certainly not improved socio-economic outcomes to
date. Many attribute the rampant inequality and environmental
degredation in the oil-rich Niger Delta to the extractive economies that
fossil fuels generate. Additionally, Nigeria has the \emph{xth} poorest
electricity service on the African continent. Despite being the
\emph{9th} largest producer of crude oil in the world, Nigeria is a net
importer of energy. A recently opened petroleum refinery, \emph{Dangote
oil..} is expected to change this. However, this refinery will process
\emph{x\%} of Nigeria's energy demand. Considering that it took \emph{x}
years for this refinery to come online, it is doubtful that complete
dependence on fossil fuels is the way forward for a population that is
expected to \emph{double?} in size over the next \emph{x} years. Rather
than simply abandoning its ``crown jewel'', Nigeria would be making a
proactive choice to preserve its natural resources, protect its
environment, bolster its economy (\emph{the government spends how much
to import refined oil}), and provide clean energy for its people if it
chose to transition to 100\% clean, renewable energy.

\hypertarget{energy-consumption-in-the-status-quo}{%
\section{Energy Consumption in the Status
Quo}\label{energy-consumption-in-the-status-quo}}

According to \emph{source}, Nigeria consumed x kWh of end-use energy in
year. This energy came from the followiing primary sources, and was
consumed by the following sectors. Considering that only x\% of people
have access to grid electricity, we know that demand is much greater
than what is currently being consumed. Additionally, we know that
generators which run on diesel fuel account for x\% of end-use energy
demand in the country\ldots{} The entire electricity generation capacity
within Nigeria is \ldots. P This takes into accounts all sectors except
for transportation.

\hypertarget{a-note-on-transportation}{%
\subsection{A Note on Transportation}\label{a-note-on-transportation}}

A clean, renewable future in Nigeria would require a re-think of
transportation entrirely. Although it has the greatest number of cars on
the African continent x, this is still x car/person. Even with this
number of cars, the traffic in Nigeria's cities is already a significant
barrier to greater economic productivity and individual well-being
\emph{(x hours)}. Given the high cost of personal vehicles relative to
income, cities in Nigeria should focus on combining efficient, electic
public transport and freight with electric last-mile shared vehicles,
which may look more along the lines of ke-ke napeps (three-wheeled
vehicles) and e-bikes than automobiles.

\hypertarget{background}{%
\section{Background}\label{background}}

This study focuses primarily on evaluatig the possibility of using solar
and wind turbine energy. I began by calculating the renewable resources
that would be available in each state. I used data from (\textbf{) to
determine the average wind speed, temperature, and solar flux (direct
normal irradiation }), available in each of the 36 Nigerian states. The
aim was to identify which states had the greatest potential for
producing renewable energy. After identifying these states, I calculated
the amount of energy that could be generated from solar and wind farms
in these locations, as well as the number of panels/turbines that would
be needed, and the amount of space that would be taken up. I structured
my calculations in such that I could evaluate how land used and items
needed would change with different configurations.

\begin{Shaded}
\begin{Highlighting}[]
\ImportTok{import}\NormalTok{ numpy }\ImportTok{as}\NormalTok{ np}
\CommentTok{\# import units as u}
\ImportTok{import}\NormalTok{ numpy }\ImportTok{as}\NormalTok{ np}
\ImportTok{import}\NormalTok{ pandas }\ImportTok{as}\NormalTok{ pd}

\ImportTok{import}\NormalTok{ rasterio}
\ImportTok{from}\NormalTok{ rasterio.plot }\ImportTok{import}\NormalTok{ show\_hist, show}
\ImportTok{import}\NormalTok{ geopandas }\ImportTok{as}\NormalTok{ gpd}
\ImportTok{import}\NormalTok{ rasterstats}
\ImportTok{from}\NormalTok{ shapely.geometry }\ImportTok{import}\NormalTok{ shape}

\ImportTok{import}\NormalTok{ plotly.graph\_objects }\ImportTok{as}\NormalTok{ go}
\ImportTok{from}\NormalTok{ plotly.subplots }\ImportTok{import}\NormalTok{ make\_subplots}
\ImportTok{from}\NormalTok{ matplotlib }\ImportTok{import}\NormalTok{ pyplot }\ImportTok{as}\NormalTok{ plt}
\end{Highlighting}
\end{Shaded}

\hypertarget{calculate-current-energy-demand}{%
\subsection{Calculate Current Energy
Demand}\label{calculate-current-energy-demand}}

TODO: use data from sources

\hypertarget{calculate-available-renewable-resource}{%
\subsection{Calculate Available Renewable
Resource}\label{calculate-available-renewable-resource}}

Using satellite data from .., I am able to see the renewable resouce
availability at relatively high resolutions of about 250 m2. In theory,
this allows me to pinpoint the exact locations where solar and wind
turbines might be located. However, given the high amounts of data, and
the discrpancy in labelling between the inputs this would be
computationally intesnive, and is left for another study. Instead, I use
the state administrative boundaries as a unifying region with which to
evaluate all the data I am bringing in. In particular, I use: - Wind
Speed (m/s): Used to to calculate the power in the wind that is
available to wind turbines - Wind Speed at Surface (TODO) - Temperature
(ºC): Average temperatures the panels will be exposed to - Solar Flux:
Available energy for solar panels

\begin{Shaded}
\begin{Highlighting}[]
\CommentTok{\# states\_path = "data/states/nga\_admbnda\_adm1\_osgof\_20161215.shp"}
\CommentTok{\# states = gpd.read\_file(states\_path)}

\CommentTok{\# data\_averages = \{\}}
\CommentTok{\# for name, path in data\_paths.items():}
\CommentTok{\#     data\_averages[name] = pd.read\_csv(f"clean\_data/\{name\}.csv", index\_col=0)}

\CommentTok{\# }\AlertTok{TODO}\CommentTok{: 2x4 subplots for resources in each states (wind, surface\_wind, flux, temperature, )}
\end{Highlighting}
\end{Shaded}

\hypertarget{calculate-solar-contribution}{%
\section{Calculate Solar
Contribution}\label{calculate-solar-contribution}}

To calculate the contribution from solar panels, I calculated the cell
temperature, T\_c, that each solar panel in a given state would be
exposed to using \texttt{Eq.\ 5.7}. I then computed C\_temp, the
correction for the cell temperature \texttt{Eq.\ 5.8}, using a
temperature coefficient of b\_ref = 0.00\ldots{} The AC power, P\_ac,
from each cell was given by \texttt{Eq\ 5.7}. I used a standard derating
factor, solar panel efficiency, and area (TODO: make table). All of
these calculations were carried out in the
\texttt{calculate\_power\_panel()} function, which, by passing in the
solar flux, ambient temperature and wind speed values averaged over the
area of each state, enabled me to calculate the average AC power output
of a single panel in each state.

In another function, \texttt{calc\_num\_panels()}, I calculated the
potential power output of solar panels. This function takes in
\texttt{p\_land}, or percentage of land, and \texttt{num\_states} or
number of states as inputs. I determine which states have the highest
average power ouput for a single solar panel, and I decide to place
panels in these states, up to \texttt{num\_states}. Within each state, I
calculate the percentage of land that will be available for solar panels
using \texttt{p\_land}. I then use the area of the input panels to
determine how many panels can be placed in this available area. I
multiply the resulting number of panels by the potential power output of
a single panel in the state to get the average power ouput. To get the
average total energy, I multiply the power by 8760 hours in a year.



\end{document}
