\documentclass{scrartcl}
\usepackage{siunitx}
\begin{document}
\hypertarget{executive-summary}{%
\section{Executive Summary}\label{executive-summary}}

\hypertarget{introduction}{%
\section{Introduction}\label{introduction}}

The world urgently needs to transition to 100\% clean, renewable energy.
This fact is particulary difficult to come to terms with in oil-rich
developing countries such as Nigeria. Here, the mining, sale, and use of
abundant fossil fuels is expected to pave the way to economic
prosperity. However, a cursory glance at the country's history reveals
that fossil fuels have certainly not improved socio-economic outcomes to
date. Many attribute the rampant inequality and environmental
degredation in the oil-rich Niger Delta to the extractive economies that
fossil fuels generate. Additionally, Nigeria has the \emph{xth} poorest
electricity service on the African continent. Despite being the
\emph{9th} largest producer of crude oil in the world, Nigeria is a net
importer of energy. A recently opened petroleum refinery, \emph{Dangote
oil..} is expected to change this. However, this refinery will process
\emph{x\%} of Nigeria's energy demand. Considering that it took \emph{x}
years for this refinery to come online, it is doubtful that complete
dependence on fossil fuels is the way forward for a population that is
expected to \emph{double?} in size over the next \emph{x} years. Rather
than simply abandoning its ``crown jewel'', Nigeria would be making a
proactive choice to preserve its natural resources, protect its
environment, bolster its economy (\emph{the government spends how much
to import refined oil}), and provide clean energy for its people if it
chose to transition to 100\% clean, renewable energy.

\hypertarget{energy-consumption-in-the-status-quo}{%
\section{Energy Consumption in the Status
Quo}\label{energy-consumption-in-the-status-quo}}

According to \emph{source}, Nigeria consumed x kWh of end-use energy in
year. This energy came from the followiing primary sources, and was
consumed by the following sectors. Considering that only x\% of people
have access to grid electricity, we know that demand is much greater
than what is currently being consumed. Additionally, we know that
generators which run on diesel fuel account for x\% of end-use energy
demand in the country\ldots{} The entire electricity generation capacity
within Nigeria is \ldots. P This takes into accounts all sectors except
for transportation.

\hypertarget{a-note-on-transportation}{%
\subsection{A Note on Transportation}\label{a-note-on-transportation}}

A clean, renewable future in Nigeria would require a re-think of
transportation entrirely. Although it has the greatest number of cars on
the African continent x, this is still x car/person. Even with this
number of cars, the traffic in Nigeria's cities is already a significant
barrier to greater economic productivity and individual well-being
\emph{(x hours)}. Given the high cost of personal vehicles relative to
income, cities in Nigeria should focus on combining efficient, electic
public transport and freight with electric last-mile shared vehicles,
which may look more along the lines of ke-ke napeps (three-wheeled
vehicles) and e-bikes than automobiles.

\hypertarget{background}{%
\section{Background}\label{background}}

This study focuses primarily on evaluatig the possibility of using solar
and wind turbine energy. I began by calculating the renewable resources
that would be available in each state. I used data from (\textbf{) to
determine the average wind speed, temperature, and solar flux (direct
normal irradiation }), available in each of the 36 Nigerian states. The
aim was to identify which states had the greatest potential for
producing renewable energy. After identifying these states, I calculated
the amount of energy that could be generated from solar and wind farms
in these locations, as well as the number of panels/turbines that would
be needed, and the amount of space that would be taken up. I structured
my calculations in such that I could evaluate how land used and items
needed would change with different configurations.

\begin{Shaded}
\begin{Highlighting}[]
\ImportTok{import}\NormalTok{ numpy }\ImportTok{as}\NormalTok{ np}
\CommentTok{\# import units as u}
\ImportTok{import}\NormalTok{ numpy }\ImportTok{as}\NormalTok{ np}
\ImportTok{import}\NormalTok{ pandas }\ImportTok{as}\NormalTok{ pd}

\ImportTok{import}\NormalTok{ rasterio}
\ImportTok{from}\NormalTok{ rasterio.plot }\ImportTok{import}\NormalTok{ show\_hist, show}
\ImportTok{import}\NormalTok{ geopandas }\ImportTok{as}\NormalTok{ gpd}
\ImportTok{import}\NormalTok{ rasterstats}
\ImportTok{from}\NormalTok{ shapely.geometry }\ImportTok{import}\NormalTok{ shape}

\ImportTok{import}\NormalTok{ plotly.graph\_objects }\ImportTok{as}\NormalTok{ go}
\ImportTok{from}\NormalTok{ plotly.subplots }\ImportTok{import}\NormalTok{ make\_subplots}
\ImportTok{from}\NormalTok{ matplotlib }\ImportTok{import}\NormalTok{ pyplot }\ImportTok{as}\NormalTok{ plt}
\end{Highlighting}
\end{Shaded}

\hypertarget{calculate-current-energy-demand}{%
\subsection{Calculate Current Energy
Demand}\label{calculate-current-energy-demand}}

TODO: use data from sources

\hypertarget{calculate-available-renewable-resource}{%
\subsection{Calculate Available Renewable
Resource}\label{calculate-available-renewable-resource}}

Using satellite data from .., I am able to see the renewable resouce
availability at relatively high resolutions of about 250 m2. In theory,
this allows me to pinpoint the exact locations where solar and wind
turbines might be located. However, given the high amounts of data, and
the discrpancy in labelling between the inputs this would be
computationally intesnive, and is left for another study. Instead, I use
the state administrative boundaries as a unifying region with which to
evaluate all the data I am bringing in. In particular, I use: - Wind
Speed (m/s): Used to to calculate the power in the wind that is
available to wind turbines - Wind Speed at Surface (TODO) - Temperature
(ºC): Average temperatures the panels will be exposed to - Solar Flux:
Available energy for solar panels * should do this with global
horizantal?

\begin{Shaded}
\begin{Highlighting}[]
\CommentTok{\# }\AlertTok{TODO}\CommentTok{: 2x4 subplots for resources in each states (wind, surface\_wind, flux, temperature, )}
\end{Highlighting}
\end{Shaded}

\hypertarget{calculate-potential-solar-contribution}{%
\section{Calculate Potential Solar
Contribution}\label{calculate-potential-solar-contribution}}

I performed the following calculations in the
\texttt{calculate\_power\_panel()} function, which, by passing in the
solar flux, ambient temperature and wind speed values averaged over the
area of each state, enabled me to calculate the average AC power output
of a single panel in each state.

To calculate the contribution from solar panels, I calculated the cell
temperature, \(T_c\), that each solar panel in a given state would be
exposed to using Equation~\ref{eq-5.9}, which is taken from the course
text, @jacobson100CleanRenewable2020a Eq 5.9. Unless otherwise stated,
all equations are referenced from the course text.
\begin{equation}\protect\hypertarget{eq-5.9}{}{ 
T_c = T_a + 0.32 * (F_{cur}/(8.91 + 2*w))
}\label{eq-5.9}\end{equation}

I then computed \(C_temp\), the correction for the cell temperature
Equation~\ref{eq-5.8}, \begin{equation}\protect\hypertarget{eq-5.8}{}{ 
C_{temp} = 1 - b_{ref} * \max\{ \min \{T_c - T_{ref}, T_{th} \}, 0 \} 
}\label{eq-5.8}\end{equation}

The AC power, \(P_ac\), from each cell was given by
Equation~\ref{eq-5.7}.\\
\begin{equation}\protect\hypertarget{eq-5.7}{}{ 
 P_{ac} = F_{cur} * A_{panel} * E_{panel} * C_{temp} * D_f
}\label{eq-5.7}\end{equation}

I used a standard values taken from the text for all constant values,
see Table~\ref{tbl-solar}.

\hypertarget{tbl-solar}{}
\begin{longtable}[]{@{}
  >{\raggedright\arraybackslash}p{(\columnwidth - 8\tabcolsep) * \real{0.1204}}
  >{\raggedright\arraybackslash}p{(\columnwidth - 8\tabcolsep) * \real{0.1481}}
  >{\raggedright\arraybackslash}p{(\columnwidth - 8\tabcolsep) * \real{0.4444}}
  >{\raggedright\arraybackslash}p{(\columnwidth - 8\tabcolsep) * \real{0.0741}}
  >{\raggedright\arraybackslash}p{(\columnwidth - 8\tabcolsep) * \real{0.2130}}@{}}
\caption{\label{tbl-solar}Terms used in equations for calculating solar
power availability}\tabularnewline
\toprule()
\begin{minipage}[b]{\linewidth}\raggedright
Equation \#
\end{minipage} & \begin{minipage}[b]{\linewidth}\raggedright
Term
\end{minipage} & \begin{minipage}[b]{\linewidth}\raggedright
Meaning
\end{minipage} & \begin{minipage}[b]{\linewidth}\raggedright
Value
\end{minipage} & \begin{minipage}[b]{\linewidth}\raggedright
Reference
\end{minipage} \\
\midrule()
\endfirsthead
\toprule()
\begin{minipage}[b]{\linewidth}\raggedright
Equation \#
\end{minipage} & \begin{minipage}[b]{\linewidth}\raggedright
Term
\end{minipage} & \begin{minipage}[b]{\linewidth}\raggedright
Meaning
\end{minipage} & \begin{minipage}[b]{\linewidth}\raggedright
Value
\end{minipage} & \begin{minipage}[b]{\linewidth}\raggedright
Reference
\end{minipage} \\
\midrule()
\endhead
\textbf{Equation~\ref{eq-5.9}} & \textbf{\(T_c\)} & \textbf{Cell
temperature} & & \textbf{Jacobson Eq. 5.9} \\
Equation~\ref{eq-5.9} & \(T_a\) & Ambient temperature (ºC) & - &
Arcsolar GIS* \\
Equation~\ref{eq-5.9} & \(F_{cur}\) & Solar flux incident on panel
(W/m2) & - & Arcsolar GIS* \\
Equation~\ref{eq-5.9} & \(w\) & Wind speed (m/s) & - & NOAA* \\
\textbf{Equation~\ref{eq-5.8}} & \textbf{\(C_{temp}\)} &
\textbf{Temperature correction } & & \textbf{Jacobson Eq. 5.8} \\
Equation~\ref{eq-5.8} & \(b_{ref}\) & Temperature coefficient (1/K) &
0.0025 & Jacobson Section.. \\
Equation~\ref{eq-5.8} & \(T_{ref}\) & Reference temperature (K) & 298.15
& Jacobson Section.. \\
Equation~\ref{eq-5.8} & \(T_{th}\) & Threshold temperature (K) & 55 &
Jacobson Section.. \\
\textbf{Equation~\ref{eq-5.7}} & \textbf{\(P_ac\)} & \textbf{Realistic
AC power output of a solar cell } & & \textbf{Jacobson Eq. 5.7} \\
Equation~\ref{eq-5.7} & \$A\_\{panel\} \$ & Solar panel area (m2) & 1.5
& Jacobson Example 5.2 \\
Equation~\ref{eq-5.7} & \$E\_\{panel\} \$ & Solar panel efficiency &
0.18 & Jacobson Example 5.2 \\
Equation~\ref{eq-5.7} & \$D\_f \$ & Derating factor & 0.864 & Jacobson
Table 5.2 \\
\bottomrule()
\end{longtable}

In another function, \texttt{calc\_num\_panels()}, I calculated a
realistic value for the potential contribution of solar to the grid,
assuming limits on land, and aiming to place the panels in the states
with the highest potential. This function takes in \texttt{p\_land}, or
percentage of land, and \texttt{num\_states} or number of states as
inputs. I determine which states have the highest average power ouput
for a single solar panel, and I decide to place panels in these states,
up to \texttt{num\_states}. Within each state, I calculate the
percentage of land that will be available for solar panels using
\texttt{p\_land}. I then use the area of the input panels to determine
how many panels can be placed in this available area. I multiply the
resulting number of panels by the potential power output of a single
panel in the state to get the average power ouput. To get the average
total energy, I multiply the power by 8760 hours in a year.

The results of perfoaming these calculations with standard values to
descibe the solar panels, and assuming \texttt{p\_land} = 0.02, and
\texttt{num\_states} = 2, are given in the table below.

\textbf{\emph{TODO}}

\hypertarget{calculate-potential-wind-contribution}{%
\section{Calculate Potential Wind
Contribution}\label{calculate-potential-wind-contribution}}

I performed the following calculations in the
\texttt{calculate\_power\_turbine()} function. I passed in the average
wind speed in a given state \(V_m\), and calculated the potential wind
power to be generated from a single turbine exposed to the ambient
conditions in that state.

I assumed standard constant values for the air density,
\(ρ = 1.23 \si{kg/m^3}\), and the diameter of the wind turbine
\(D_t = 127 \si{m}\). I calculated the swept area of the turbine,
\(A_t\) (m\^{}2), using Equation~\ref{eq-6.7} (Eq 6.7).
\begin{equation}\protect\hypertarget{eq-6.7}{}{ 
   A_t = π * D_t^2 / 4
}\label{eq-6.7}\end{equation}

I then calculated the power available in the wind, \(P_m\) (W), using
the average wind speed in a given state, \(V_m\) and
Equation~\ref{eq-6.11} (Eq 6.11)
\begin{equation}\protect\hypertarget{eq-6.11}{}{ 
   P_m = (6/ π) * (0.5) * rho * A_t * V_m^3
}\label{eq-6.11}\end{equation}

Once I had \(P_m\) in each state, I determined what would be an
appropriate power rating, \(P_r\) for a turbine in that state by
rounding the value of \(P_m\) up to the next 500. I then calculated the
capacity factor for a turbine with this rating using
Equation~\ref{eq-6.28} (Eq 6.28).
\begin{equation}\protect\hypertarget{eq-6.28}{}{ 
   CF = 0.087 * V_m - ((P_r)/(D_t^2))
}\label{eq-6.28}\end{equation}

I then calculated the expected power from a single turbine in a given
state by multiplying the capacity factor by the rated power in
Equation~\ref{eq-6.24} (Eq. 6.24).
\begin{equation}\protect\hypertarget{eq-6.24}{}{ 
   P_t = P_r * CF
}\label{eq-6.24}\end{equation}

I accounted for losses that would occur in the system due to curtailment
using the following equations, Equation~\ref{eq-6.31} and
Equation~\ref{eq-6.31b} (Eq 6.31). The values of the losses I used are
described in Table~\ref{tbl-wind-losses}.
\begin{equation}\protect\hypertarget{eq-6.31}{}{ 
  L_t = (1 - (p_{L_td} * p_{L_d} * p_{L_c} * p_{L_a} ))
}\label{eq-6.31}\end{equation}

\begin{equation}\protect\hypertarget{eq-6.31b}{}{ 
  P_{t \ after \ losses} = P_t * (1 - L_t)
}\label{eq-6.31b}\end{equation}

\hypertarget{tbl-wind-losses}{}
\begin{longtable}[]{@{}
  >{\raggedright\arraybackslash}p{(\columnwidth - 6\tabcolsep) * \real{0.1143}}
  >{\raggedright\arraybackslash}p{(\columnwidth - 6\tabcolsep) * \real{0.3429}}
  >{\raggedright\arraybackslash}p{(\columnwidth - 6\tabcolsep) * \real{0.0762}}
  >{\raggedright\arraybackslash}p{(\columnwidth - 6\tabcolsep) * \real{0.4667}}@{}}
\caption{\label{tbl-wind-losses}Values of losses used in calculating
turbine power}\tabularnewline
\toprule()
\begin{minipage}[b]{\linewidth}\raggedright
Variable
\end{minipage} & \begin{minipage}[b]{\linewidth}\raggedright
Loss Type
\end{minipage} & \begin{minipage}[b]{\linewidth}\raggedright
Value
\end{minipage} & \begin{minipage}[b]{\linewidth}\raggedright
Reference
\end{minipage} \\
\midrule()
\endfirsthead
\toprule()
\begin{minipage}[b]{\linewidth}\raggedright
Variable
\end{minipage} & \begin{minipage}[b]{\linewidth}\raggedright
Loss Type
\end{minipage} & \begin{minipage}[b]{\linewidth}\raggedright
Value
\end{minipage} & \begin{minipage}[b]{\linewidth}\raggedright
Reference
\end{minipage} \\
\midrule()
\endhead
\(p_{L_td}\) & Transmission and distribution loss & 16.1 &
@jacobson100CleanRenewable2020a Table 6.3 \\
\(p_{L_d}\) & Downtime loss & 1.6 & @jacobson100CleanRenewable2020a
Section 6.6.6.2 \\
\(p_{L_c}\) & Curtailment loss & 0 & Assume storage is available \\
\(p_{L_a}\) & Array loss & 0.2 & @jacobson100CleanRenewable2020a Figure
6.25b \\
\bottomrule()
\end{longtable}

To calculate the potential contribution of wind energy to the grid, I
followed a similar procedure to what was described in the previous
section. I passed in a percentage of land \texttt{p\_land}, and desired
\texttt{num\_states} into \texttt{calc\_num\_states} to determine how
much energy could be produced under reasonable limits of land use. I
determined how many wind turbines could be realistically placed on this
land using an installed power density of \(19.8 \si{W/m^2}\) which was
refrenced from Table 6.4 in @jacobson100CleanRenewable2020a. This is a
value taken from a study of onshore wind turbines in Europe.

The results of perfoaming these calculations with standard values to
descibe the solar panels, and assuming \texttt{p\_land} = 0.02, and
\texttt{num\_states} = 2, are given in the table below.

\textbf{TODO: calculate spacing and footing}
\end{document}
