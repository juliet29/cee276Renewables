% Options for packages loaded elsewhere
\PassOptionsToPackage{unicode}{hyperref}
\PassOptionsToPackage{hyphens}{url}
\PassOptionsToPackage{dvipsnames,svgnames,x11names}{xcolor}
%
\documentclass[
  letterpaper,
  DIV=11,
  numbers=noendperiod]{scrartcl}

\usepackage{amsmath,amssymb}
\usepackage{lmodern}
\usepackage{iftex}
\ifPDFTeX
  \usepackage[T1]{fontenc}
  \usepackage[utf8]{inputenc}
  \usepackage{textcomp} % provide euro and other symbols
\else % if luatex or xetex
  \usepackage{unicode-math}
  \defaultfontfeatures{Scale=MatchLowercase}
  \defaultfontfeatures[\rmfamily]{Ligatures=TeX,Scale=1}
\fi
% Use upquote if available, for straight quotes in verbatim environments
\IfFileExists{upquote.sty}{\usepackage{upquote}}{}
\IfFileExists{microtype.sty}{% use microtype if available
  \usepackage[]{microtype}
  \UseMicrotypeSet[protrusion]{basicmath} % disable protrusion for tt fonts
}{}
\makeatletter
\@ifundefined{KOMAClassName}{% if non-KOMA class
  \IfFileExists{parskip.sty}{%
    \usepackage{parskip}
  }{% else
    \setlength{\parindent}{0pt}
    \setlength{\parskip}{6pt plus 2pt minus 1pt}}
}{% if KOMA class
  \KOMAoptions{parskip=half}}
\makeatother
\usepackage{xcolor}
\setlength{\emergencystretch}{3em} % prevent overfull lines
\setcounter{secnumdepth}{-\maxdimen} % remove section numbering
% Make \paragraph and \subparagraph free-standing
\ifx\paragraph\undefined\else
  \let\oldparagraph\paragraph
  \renewcommand{\paragraph}[1]{\oldparagraph{#1}\mbox{}}
\fi
\ifx\subparagraph\undefined\else
  \let\oldsubparagraph\subparagraph
  \renewcommand{\subparagraph}[1]{\oldsubparagraph{#1}\mbox{}}
\fi

\usepackage{color}
\usepackage{fancyvrb}
\newcommand{\VerbBar}{|}
\newcommand{\VERB}{\Verb[commandchars=\\\{\}]}
\DefineVerbatimEnvironment{Highlighting}{Verbatim}{commandchars=\\\{\}}
% Add ',fontsize=\small' for more characters per line
\usepackage{framed}
\definecolor{shadecolor}{RGB}{241,243,245}
\newenvironment{Shaded}{\begin{snugshade}}{\end{snugshade}}
\newcommand{\AlertTok}[1]{\textcolor[rgb]{0.68,0.00,0.00}{#1}}
\newcommand{\AnnotationTok}[1]{\textcolor[rgb]{0.37,0.37,0.37}{#1}}
\newcommand{\AttributeTok}[1]{\textcolor[rgb]{0.40,0.45,0.13}{#1}}
\newcommand{\BaseNTok}[1]{\textcolor[rgb]{0.68,0.00,0.00}{#1}}
\newcommand{\BuiltInTok}[1]{\textcolor[rgb]{0.00,0.23,0.31}{#1}}
\newcommand{\CharTok}[1]{\textcolor[rgb]{0.13,0.47,0.30}{#1}}
\newcommand{\CommentTok}[1]{\textcolor[rgb]{0.37,0.37,0.37}{#1}}
\newcommand{\CommentVarTok}[1]{\textcolor[rgb]{0.37,0.37,0.37}{\textit{#1}}}
\newcommand{\ConstantTok}[1]{\textcolor[rgb]{0.56,0.35,0.01}{#1}}
\newcommand{\ControlFlowTok}[1]{\textcolor[rgb]{0.00,0.23,0.31}{#1}}
\newcommand{\DataTypeTok}[1]{\textcolor[rgb]{0.68,0.00,0.00}{#1}}
\newcommand{\DecValTok}[1]{\textcolor[rgb]{0.68,0.00,0.00}{#1}}
\newcommand{\DocumentationTok}[1]{\textcolor[rgb]{0.37,0.37,0.37}{\textit{#1}}}
\newcommand{\ErrorTok}[1]{\textcolor[rgb]{0.68,0.00,0.00}{#1}}
\newcommand{\ExtensionTok}[1]{\textcolor[rgb]{0.00,0.23,0.31}{#1}}
\newcommand{\FloatTok}[1]{\textcolor[rgb]{0.68,0.00,0.00}{#1}}
\newcommand{\FunctionTok}[1]{\textcolor[rgb]{0.28,0.35,0.67}{#1}}
\newcommand{\ImportTok}[1]{\textcolor[rgb]{0.00,0.46,0.62}{#1}}
\newcommand{\InformationTok}[1]{\textcolor[rgb]{0.37,0.37,0.37}{#1}}
\newcommand{\KeywordTok}[1]{\textcolor[rgb]{0.00,0.23,0.31}{#1}}
\newcommand{\NormalTok}[1]{\textcolor[rgb]{0.00,0.23,0.31}{#1}}
\newcommand{\OperatorTok}[1]{\textcolor[rgb]{0.37,0.37,0.37}{#1}}
\newcommand{\OtherTok}[1]{\textcolor[rgb]{0.00,0.23,0.31}{#1}}
\newcommand{\PreprocessorTok}[1]{\textcolor[rgb]{0.68,0.00,0.00}{#1}}
\newcommand{\RegionMarkerTok}[1]{\textcolor[rgb]{0.00,0.23,0.31}{#1}}
\newcommand{\SpecialCharTok}[1]{\textcolor[rgb]{0.37,0.37,0.37}{#1}}
\newcommand{\SpecialStringTok}[1]{\textcolor[rgb]{0.13,0.47,0.30}{#1}}
\newcommand{\StringTok}[1]{\textcolor[rgb]{0.13,0.47,0.30}{#1}}
\newcommand{\VariableTok}[1]{\textcolor[rgb]{0.07,0.07,0.07}{#1}}
\newcommand{\VerbatimStringTok}[1]{\textcolor[rgb]{0.13,0.47,0.30}{#1}}
\newcommand{\WarningTok}[1]{\textcolor[rgb]{0.37,0.37,0.37}{\textit{#1}}}

\providecommand{\tightlist}{%
  \setlength{\itemsep}{0pt}\setlength{\parskip}{0pt}}\usepackage{longtable,booktabs,array}
\usepackage{calc} % for calculating minipage widths
% Correct order of tables after \paragraph or \subparagraph
\usepackage{etoolbox}
\makeatletter
\patchcmd\longtable{\par}{\if@noskipsec\mbox{}\fi\par}{}{}
\makeatother
% Allow footnotes in longtable head/foot
\IfFileExists{footnotehyper.sty}{\usepackage{footnotehyper}}{\usepackage{footnote}}
\makesavenoteenv{longtable}
\usepackage{graphicx}
\makeatletter
\def\maxwidth{\ifdim\Gin@nat@width>\linewidth\linewidth\else\Gin@nat@width\fi}
\def\maxheight{\ifdim\Gin@nat@height>\textheight\textheight\else\Gin@nat@height\fi}
\makeatother
% Scale images if necessary, so that they will not overflow the page
% margins by default, and it is still possible to overwrite the defaults
% using explicit options in \includegraphics[width, height, ...]{}
\setkeys{Gin}{width=\maxwidth,height=\maxheight,keepaspectratio}
% Set default figure placement to htbp
\makeatletter
\def\fps@figure{htbp}
\makeatother

\KOMAoption{captions}{tableheading}
\makeatletter
\makeatother
\makeatletter
\makeatother
\makeatletter
\@ifpackageloaded{caption}{}{\usepackage{caption}}
\AtBeginDocument{%
\ifdefined\contentsname
  \renewcommand*\contentsname{Table of contents}
\else
  \newcommand\contentsname{Table of contents}
\fi
\ifdefined\listfigurename
  \renewcommand*\listfigurename{List of Figures}
\else
  \newcommand\listfigurename{List of Figures}
\fi
\ifdefined\listtablename
  \renewcommand*\listtablename{List of Tables}
\else
  \newcommand\listtablename{List of Tables}
\fi
\ifdefined\figurename
  \renewcommand*\figurename{Figure}
\else
  \newcommand\figurename{Figure}
\fi
\ifdefined\tablename
  \renewcommand*\tablename{Table}
\else
  \newcommand\tablename{Table}
\fi
}
\@ifpackageloaded{float}{}{\usepackage{float}}
\floatstyle{ruled}
\@ifundefined{c@chapter}{\newfloat{codelisting}{h}{lop}}{\newfloat{codelisting}{h}{lop}[chapter]}
\floatname{codelisting}{Listing}
\newcommand*\listoflistings{\listof{codelisting}{List of Listings}}
\makeatother
\makeatletter
\@ifpackageloaded{caption}{}{\usepackage{caption}}
\@ifpackageloaded{subcaption}{}{\usepackage{subcaption}}
\makeatother
\makeatletter
\@ifpackageloaded{tcolorbox}{}{\usepackage[many]{tcolorbox}}
\makeatother
\makeatletter
\@ifundefined{shadecolor}{\definecolor{shadecolor}{rgb}{.97, .97, .97}}
\makeatother
\makeatletter
\makeatother
\ifLuaTeX
  \usepackage{selnolig}  % disable illegal ligatures
\fi
\IfFileExists{bookmark.sty}{\usepackage{bookmark}}{\usepackage{hyperref}}
\IfFileExists{xurl.sty}{\usepackage{xurl}}{} % add URL line breaks if available
\urlstyle{same} % disable monospaced font for URLs
\hypersetup{
  pdftitle={Renewables Homework 3},
  colorlinks=true,
  linkcolor={blue},
  filecolor={Maroon},
  citecolor={Blue},
  urlcolor={Blue},
  pdfcreator={LaTeX via pandoc}}

\title{Renewables Homework 3}
\author{}
\date{}

\begin{document}
\maketitle
\ifdefined\Shaded\renewenvironment{Shaded}{\begin{tcolorbox}[borderline west={3pt}{0pt}{shadecolor}, enhanced, boxrule=0pt, frame hidden, sharp corners, breakable, interior hidden]}{\end{tcolorbox}}\fi

\begin{Shaded}
\begin{Highlighting}[]
\ImportTok{import}\NormalTok{ numpy }\ImportTok{as}\NormalTok{ np}
\ImportTok{import}\NormalTok{ units }\ImportTok{as}\NormalTok{ u}
\ImportTok{import}\NormalTok{ numpy }\ImportTok{as}\NormalTok{ np}
\ImportTok{from}\NormalTok{ IPython.display }\ImportTok{import}\NormalTok{ display, Markdown, Latex}
\end{Highlighting}
\end{Shaded}

\hypertarget{chapter-6}{%
\section{Chapter 6}\label{chapter-6}}

\hypertarget{section}{%
\subsection{6.2.}\label{section}}

\textbf{\emph{Under which, if either, condition is the lift-to-drag
ratio of a wind turbine airfoil higher? At 10 m height where the air
density is 1.23 kg/m3 and the wind speed is 5 m/s, or at 100 m height,
where the air density is 1.21 kg/m3 and the wind speed is 9 m/s?}}

\textbf{\emph{Also, under which condition is the lift greater? Assume
the blade length is 60 m, the blade width is an average of 1 m, the drag
coefficient is 0.01, and the lift coefficient is 1.0.}}

\begin{Shaded}
\begin{Highlighting}[]
\KeywordTok{def}\NormalTok{ calc\_lift\_or\_drag(C, rho, v, Af,):}
    \ControlFlowTok{return}\NormalTok{ C}\OperatorTok{*}\NormalTok{rho}\OperatorTok{*}\NormalTok{Af}\OperatorTok{*}\NormalTok{v}


\NormalTok{C\_lift }\OperatorTok{=} \DecValTok{1}
\NormalTok{C\_drag }\OperatorTok{=} \FloatTok{0.01} 

\NormalTok{rho\_a }\OperatorTok{=} \FloatTok{1.23} \OperatorTok{*}\NormalTok{ u.kg }\OperatorTok{/}\NormalTok{ u.m}\OperatorTok{**}\DecValTok{3}
\NormalTok{v\_a }\OperatorTok{=} \DecValTok{5} \OperatorTok{*}\NormalTok{ u.m}\OperatorTok{/}\NormalTok{ u.s}

\NormalTok{rho\_b }\OperatorTok{=} \FloatTok{1.21} \OperatorTok{*}\NormalTok{ u.kg }\OperatorTok{/}\NormalTok{ u.m}\OperatorTok{**}\DecValTok{3}
\NormalTok{v\_b }\OperatorTok{=} \DecValTok{8} \OperatorTok{*}\NormalTok{ u.m}\OperatorTok{/}\NormalTok{ u.s}

\NormalTok{blade\_area }\OperatorTok{=} \DecValTok{60} \OperatorTok{*}\NormalTok{ u.m }\OperatorTok{*} \DecValTok{1} \OperatorTok{*}\NormalTok{ u.m }


\NormalTok{lift\_a }\OperatorTok{=}\NormalTok{ calc\_lift\_or\_drag(C}\OperatorTok{=}\NormalTok{C\_lift, rho}\OperatorTok{=}\NormalTok{rho\_a, v}\OperatorTok{=}\NormalTok{v\_a, Af}\OperatorTok{=}\NormalTok{blade\_area)}
\NormalTok{lift\_b }\OperatorTok{=}\NormalTok{ calc\_lift\_or\_drag(C}\OperatorTok{=}\NormalTok{C\_lift, rho}\OperatorTok{=}\NormalTok{rho\_b, v}\OperatorTok{=}\NormalTok{v\_b, Af}\OperatorTok{=}\NormalTok{blade\_area)}

\NormalTok{drag\_a }\OperatorTok{=}\NormalTok{ calc\_lift\_or\_drag(C}\OperatorTok{=}\NormalTok{C\_drag, rho}\OperatorTok{=}\NormalTok{rho\_a, v}\OperatorTok{=}\NormalTok{v\_a, Af}\OperatorTok{=}\NormalTok{blade\_area)}
\NormalTok{drag\_b }\OperatorTok{=}\NormalTok{ calc\_lift\_or\_drag(C}\OperatorTok{=}\NormalTok{C\_drag, rho}\OperatorTok{=}\NormalTok{rho\_a, v}\OperatorTok{=}\NormalTok{v\_a, Af}\OperatorTok{=}\NormalTok{blade\_area)}

\NormalTok{display(Markdown(}\SpecialStringTok{f"""}
\SpecialStringTok{For case A, the lift{-}to{-}drag ratio is }\SpecialCharTok{\{}\NormalTok{lift\_a}\OperatorTok{/}\NormalTok{drag\_a}\SpecialCharTok{:}\NormalTok{P}\OperatorTok{\textasciitilde{}}\SpecialCharTok{\}}\SpecialStringTok{, while for case B it is }\SpecialCharTok{\{}\NormalTok{np}\SpecialCharTok{.}\BuiltInTok{round}\NormalTok{(lift\_b}\OperatorTok{/}\NormalTok{drag\_b,}\DecValTok{0}\NormalTok{)}\SpecialCharTok{:}\NormalTok{P}\OperatorTok{\textasciitilde{}}\SpecialCharTok{\}}\SpecialStringTok{. Case B has the higher ratio. }

\SpecialStringTok{In case A, the lift is }\SpecialCharTok{\{}\NormalTok{lift\_a}\SpecialCharTok{:}\NormalTok{P}\OperatorTok{\textasciitilde{}}\SpecialCharTok{\}}\SpecialStringTok{, but in case B it is }\SpecialCharTok{\{}\NormalTok{lift\_b}\SpecialCharTok{:}\NormalTok{P}\OperatorTok{\textasciitilde{}}\SpecialCharTok{\}}\SpecialStringTok{. Case B has the greater lift. }
\SpecialStringTok{"""}\NormalTok{))}
\end{Highlighting}
\end{Shaded}

For case A, the lift-to-drag ratio is 100.0, while for case B it is
157.0. Case B has the higher ratio.

In case A, the lift is 369.0 kg/s, but in case B it is 580.8 kg/s. Case
B has the greater lift.

\hypertarget{section-1}{%
\subsection{6.4.}\label{section-1}}

\textbf{\emph{If a wind turbine's blade diameter is 100 m, calculate the
kinetic energy passing through the blade's swept area over 1 hour if the
wind speed is 10 m/s. Assume the air density is 1.21 kg/m3. Also
determine the power in the wind.}}

\begin{Shaded}
\begin{Highlighting}[]
\NormalTok{Dt }\OperatorTok{=} \DecValTok{100} \OperatorTok{*}\NormalTok{ u.m}
\NormalTok{t }\OperatorTok{=} \DecValTok{1} \OperatorTok{*}\NormalTok{ u.hour }
\NormalTok{v }\OperatorTok{=} \DecValTok{10} \OperatorTok{*}\NormalTok{ u.m }\OperatorTok{/}\NormalTok{ u.s }
\NormalTok{rho }\OperatorTok{=} \FloatTok{1.21} \OperatorTok{*}\NormalTok{ u.kg }\OperatorTok{/}\NormalTok{ u.m}\OperatorTok{**}\DecValTok{3}

\NormalTok{x }\OperatorTok{=}\NormalTok{ v }\OperatorTok{*}\NormalTok{ t }
\NormalTok{At }\OperatorTok{=}\NormalTok{ np.pi }\OperatorTok{*}\NormalTok{ (}\FloatTok{0.5}\OperatorTok{*}\NormalTok{Dt)}\OperatorTok{**}\DecValTok{2} \CommentTok{\# A = pi * r\^{}2 }
\NormalTok{Ma }\OperatorTok{=}\NormalTok{ rho }\OperatorTok{*}\NormalTok{ At }\OperatorTok{*}\NormalTok{ x }

\NormalTok{KE }\OperatorTok{=} \FloatTok{0.5} \OperatorTok{*}\NormalTok{ Ma }\OperatorTok{*}\NormalTok{ v}\OperatorTok{**}\DecValTok{2}

\NormalTok{P }\OperatorTok{=} \FloatTok{0.5} \OperatorTok{*}\NormalTok{ rho }\OperatorTok{*}\NormalTok{ At }\OperatorTok{*}\NormalTok{ v}\OperatorTok{**}\DecValTok{3}


\NormalTok{display(Markdown(}\SpecialStringTok{f"""}
\SpecialStringTok{The kinetic energy is }\SpecialCharTok{\{}\NormalTok{KE}\SpecialCharTok{.}\NormalTok{to(}\StringTok{"megajoules"}\NormalTok{)}\SpecialCharTok{:}\FloatTok{.2}\ErrorTok{E}\OperatorTok{\textasciitilde{}}\NormalTok{P}\SpecialCharTok{\}}\SpecialStringTok{, and the power in the wind is }\SpecialCharTok{\{}\NormalTok{P}\SpecialCharTok{.}\NormalTok{to(}\StringTok{"kilowatt"}\NormalTok{)}\SpecialCharTok{:}\FloatTok{.2}\ErrorTok{E}\OperatorTok{\textasciitilde{}}\NormalTok{P}\SpecialCharTok{\}}\SpecialStringTok{.}
\SpecialStringTok{"""}\NormalTok{))}
\end{Highlighting}
\end{Shaded}

The kinetic energy is 1.71E+04 MJ, and the power in the wind is 4.75E+03
kW.

\hypertarget{section-2}{%
\subsection{6.5.}\label{section-2}}

\textbf{\emph{If the wind speed at 100 m height above the ground is 8
m/s and that at 10 m height is 5 m/s, calculate the power law
coefficient for this profile. Using the coefficient, estimate the wind
speed at 200 m above the ground.}}

\begin{Shaded}
\begin{Highlighting}[]
\CommentTok{\# hello whats up }
\NormalTok{v100 }\OperatorTok{=} \DecValTok{8} \OperatorTok{*}\NormalTok{ u.m }\OperatorTok{/}\NormalTok{ u.s }
\NormalTok{h100 }\OperatorTok{=} \DecValTok{100} \OperatorTok{*}\NormalTok{ u.m }

\NormalTok{v10 }\OperatorTok{=} \DecValTok{5} \OperatorTok{*}\NormalTok{ u.m }\OperatorTok{/}\NormalTok{ u.s }
\NormalTok{h10 }\OperatorTok{=} \DecValTok{10} \OperatorTok{*}\NormalTok{ u.m }

\CommentTok{\# v100 = v10 * (h100/h10)\^{}alpha }
\CommentTok{\# log10(v100/v10) = log10((h100/h10)\^{}alpha) {-}{-}\textgreater{}  h100/h10 = 10 }
\CommentTok{\# log10(v100/v10) = log10((10)\^{}alpha)}
\CommentTok{\# log10(v100/v10) = alpha}

\NormalTok{alpha }\OperatorTok{=}\NormalTok{ np.log10(v100}\OperatorTok{/}\NormalTok{v10)}

\NormalTok{h200 }\OperatorTok{=} \DecValTok{200} \OperatorTok{*}\NormalTok{ u.m}
\NormalTok{v200 }\OperatorTok{=}\NormalTok{ v10 }\OperatorTok{*}\NormalTok{ (h200}\OperatorTok{/}\NormalTok{h10)}\OperatorTok{**}\NormalTok{alpha}

\NormalTok{display(Markdown(}\SpecialStringTok{f"""}
\SpecialStringTok{Alpha is  }\SpecialCharTok{\{}\NormalTok{alpha}\SpecialCharTok{:}\FloatTok{.2}\ErrorTok{f}\OperatorTok{\textasciitilde{}}\NormalTok{P}\SpecialCharTok{\}}\SpecialStringTok{, and the wind speed at }\SpecialCharTok{\{}\NormalTok{h200}\SpecialCharTok{:}\FloatTok{.0}\ErrorTok{f}\OperatorTok{\textasciitilde{}}\NormalTok{P}\SpecialCharTok{\}}\SpecialStringTok{ is }\SpecialCharTok{\{}\NormalTok{v200}\SpecialCharTok{:}\FloatTok{.2}\ErrorTok{f}\OperatorTok{\textasciitilde{}}\NormalTok{P}\SpecialCharTok{\}}\SpecialStringTok{.}
\SpecialStringTok{"""}\NormalTok{))}
\end{Highlighting}
\end{Shaded}

Alpha is 0.20, and the wind speed at 200 m is 9.22 m/s.

\hypertarget{section-3}{%
\subsection{6.9.}\label{section-3}}

\textbf{\emph{Using the empirical capacity factor equation, estimate the
capacity factors of a 5-MW turbine with a 126-m blade diameter and of a
1.5-MW turbine with a 77-m blade diameter when the mean wind speed
(assuming a Rayleigh distribution) is 8 m/s in both cases. What is the
annual energy output in both cases (assuming a non-leap year)?}}

\begin{Shaded}
\begin{Highlighting}[]
\CommentTok{\# no units, equation is not empirical}

\NormalTok{Vm }\OperatorTok{=} \DecValTok{8} \CommentTok{\# * u.m / u.s}
\NormalTok{st\_year\_hours }\OperatorTok{=} \DecValTok{8760} \CommentTok{\# * u.hour, standard year hours }

\CommentTok{\# turbineA = \{"Pr": 5000 * u.kilowatt, "D": 126 * u.m\}}
\CommentTok{\# turbineB = \{"Pr": 1500 * u.kilowatt, "D": 77 * u.m\}}
\NormalTok{turbineA }\OperatorTok{=}\NormalTok{ \{}\StringTok{"Pr"}\NormalTok{: }\DecValTok{5000}\NormalTok{ , }\StringTok{"D"}\NormalTok{: }\DecValTok{126}\NormalTok{ \}}
\NormalTok{turbineB }\OperatorTok{=}\NormalTok{ \{}\StringTok{"Pr"}\NormalTok{: }\DecValTok{1500}\NormalTok{, }\StringTok{"D"}\NormalTok{: }\DecValTok{77}\NormalTok{ \}}

\KeywordTok{def}\NormalTok{ calc\_CF\_and\_energy(t, Vm}\OperatorTok{=}\NormalTok{Vm, hours}\OperatorTok{=}\NormalTok{st\_year\_hours):}
\NormalTok{    CF }\OperatorTok{=} \FloatTok{0.087} \OperatorTok{*}\NormalTok{ Vm }\OperatorTok{{-}}\NormalTok{ (t[}\StringTok{"Pr"}\NormalTok{]}\OperatorTok{/}\NormalTok{ t[}\StringTok{"D"}\NormalTok{]}\OperatorTok{**}\DecValTok{2}\NormalTok{)}
\NormalTok{    Et }\OperatorTok{=}\NormalTok{ t[}\StringTok{"Pr"}\NormalTok{] }\OperatorTok{*}\NormalTok{ CF }\OperatorTok{*}\NormalTok{ hours }\OperatorTok{*}\NormalTok{ u.j}
    \ControlFlowTok{return}\NormalTok{ CF, Et.to(}\StringTok{"megajoule"}\NormalTok{)}

\NormalTok{A }\OperatorTok{=}\NormalTok{ calc\_CF\_and\_energy(turbineA)}
\NormalTok{B }\OperatorTok{=}\NormalTok{ calc\_CF\_and\_energy(turbineB)}


\NormalTok{display(Markdown(}\SpecialStringTok{f"""}
\SpecialStringTok{For the 5{-}MW turbine, the capacity factor is   }\SpecialCharTok{\{}\NormalTok{A[}\DecValTok{0}\NormalTok{]}\SpecialCharTok{:.2f\}}\SpecialStringTok{, and the annual energy output is  }\SpecialCharTok{\{}\NormalTok{A[}\DecValTok{1}\NormalTok{]}\SpecialCharTok{:}\FloatTok{.0}\ErrorTok{f}\OperatorTok{\textasciitilde{}}\NormalTok{P}\SpecialCharTok{\}}\SpecialStringTok{.}

\SpecialStringTok{For the 1.5{-}MW turbine, the capacity factor is   }\SpecialCharTok{\{}\NormalTok{B[}\DecValTok{0}\NormalTok{]}\SpecialCharTok{:.2f\}}\SpecialStringTok{, and the annual energy output is  }\SpecialCharTok{\{}\NormalTok{B[}\DecValTok{1}\NormalTok{]}\SpecialCharTok{:}\FloatTok{.0}\ErrorTok{f}\OperatorTok{\textasciitilde{}}\NormalTok{P}\SpecialCharTok{\}}\SpecialStringTok{.}
\SpecialStringTok{"""}\NormalTok{))}
\end{Highlighting}
\end{Shaded}

For the 5-MW turbine, the capacity factor is 0.38, and the annual energy
output is 17 MJ.

For the 1.5-MW turbine, the capacity factor is 0.44, and the annual
energy output is 6 MJ.

\hypertarget{section-4}{%
\subsection{6.11.}\label{section-4}}

\textbf{\emph{Calculate the reduction in electricity generation that is
possible in Haiti, without reducing end-use electricity availability, if
the transmission and distribution loss is reduced by 5 percentage points
from its 2014 value.}}

\begin{Shaded}
\begin{Highlighting}[]
\NormalTok{del\_Ltd }\OperatorTok{=} \DecValTok{5} \CommentTok{\# percentage }
\NormalTok{Ltd }\OperatorTok{=} \FloatTok{60.1}
\NormalTok{del\_Gelec }\OperatorTok{=}\NormalTok{ (}\DecValTok{100} \OperatorTok{*}\NormalTok{ del\_Ltd )}\OperatorTok{/}\NormalTok{(}\DecValTok{100} \OperatorTok{{-}}\NormalTok{ Ltd }\OperatorTok{+}\NormalTok{ del\_Ltd)}

\NormalTok{display(Markdown(}\SpecialStringTok{f"""}
\SpecialStringTok{Haiti can reduce electricity generation by }\SpecialCharTok{\{}\NormalTok{del\_Gelec}\SpecialCharTok{:.2f\}}\SpecialStringTok{\%  by reducing its transmission and distribution losses.}
\SpecialStringTok{"""}\NormalTok{))}
\end{Highlighting}
\end{Shaded}

Haiti can reduce electricity generation by 11.14\% by reducing its
transmission and distribution losses.

\hypertarget{section-5}{%
\subsection{6.14.}\label{section-5}}

\textbf{\emph{Explain the difference between footprint and spacing of an
energy technology. What are some of the uses of the spacing?}}

The footprint is the physical area on the topsoil or water that is
touched by an energy technology. Spacing is the area that is essentially
``in between'' the energy technologies. For instance, wind turbines
require a certain spacing for optimal energy production and to prevent
harm to the surrounding locations and each other. Spacing can be used
for agriculture, other energy technologies like solar arrays, or can
exist as forests/open space.

\hypertarget{section-6}{%
\subsection{6.16.}\label{section-6}}

\textbf{\emph{How many 5-MW onshore wind turbines with a rotor diameter
of 126 m operating in a mean annual wind speed of 7.5 m/s are needed to
power the U.S. on-road vehicle fleet consisting of battery-electric (BE)
vehicles if the end-use energy required to run such a fleet is Ev = 1.15
× 1012 kWh/y (2017) and the plug-to-wheel efficiency of an electric
vehicle is ηe = 0.85? Assume the system efficiency of each wind turbine
is ηt = 0.9. Hint: First determine the total electrical energy required
to run the fleet by dividing the end-use energy required to run vehicles
by the plug-to-wheel efficiency.}}

\begin{Shaded}
\begin{Highlighting}[]
\CommentTok{\# {-}{-}{-} electric vehicle fleet }
\NormalTok{Ev }\OperatorTok{=} \FloatTok{1.15e12} \OperatorTok{*}\NormalTok{ u.kwh }\OperatorTok{/}\NormalTok{ u.year }\CommentTok{\# end use energy of battery electric vehicle fleet }
\NormalTok{eta\_e }\OperatorTok{=} \FloatTok{0.85} \CommentTok{\# plug{-}to{-}wheel efficiency of an electric vehicle}

\NormalTok{Ev\_total }\OperatorTok{=}\NormalTok{ (Ev}\OperatorTok{/}\NormalTok{eta\_e).to(}\StringTok{"megawatt * hour / year"}\NormalTok{) }\OperatorTok{*}\NormalTok{ u.year}
\NormalTok{Ev\_total\_joules }\OperatorTok{=}\NormalTok{ (Ev\_total).to(}\StringTok{"megajoule"}\NormalTok{)}

\CommentTok{\# {-}{-} wind turbines }
\CommentTok{\# See function definition in problem 6.9. The turbine described here has the same characteristics of turbine A in that problekm =\textgreater{} 5{-}MW and 126m diameter }
\NormalTok{Vm2 }\OperatorTok{=} \FloatTok{7.5} \CommentTok{\# u.m / u.s}
\NormalTok{CF, Et }\OperatorTok{=}\NormalTok{ calc\_CF\_and\_energy(turbineA, Vm}\OperatorTok{=}\NormalTok{Vm2)}

\NormalTok{eta\_t }\OperatorTok{=} \FloatTok{0.9} \CommentTok{\# turbine efficiency }
\NormalTok{Et\_available }\OperatorTok{=}\NormalTok{ Et }\OperatorTok{*}\NormalTok{ eta\_t }

\NormalTok{n\_turbines }\OperatorTok{=}\NormalTok{ Ev\_total\_joules }\OperatorTok{/}\NormalTok{ Et\_available }

\NormalTok{display(Markdown(}\SpecialStringTok{f"""}
\SpecialStringTok{Accounting for the plug{-}to{-}wheel efficiency of an electric vehicle, the total energy needed to run the fleet will be }\SpecialCharTok{\{}\NormalTok{Ev\_total}\SpecialCharTok{:}\FloatTok{.2}\ErrorTok{E}\OperatorTok{\textasciitilde{}}\NormalTok{P}\SpecialCharTok{\}}\SpecialStringTok{ per year, or }\SpecialCharTok{\{}\NormalTok{Ev\_total\_joules}\SpecialCharTok{:}\FloatTok{.2}\ErrorTok{E}\OperatorTok{\textasciitilde{}}\NormalTok{P}\SpecialCharTok{\}}\SpecialStringTok{.}
\SpecialStringTok{"""}\NormalTok{))}

\NormalTok{display(Markdown(}\SpecialStringTok{f"""}
\SpecialStringTok{After accounting for the system efficiency of each turbine, the energy that will be availabe from each will be }\SpecialCharTok{\{}\NormalTok{Et\_available}\SpecialCharTok{:}\FloatTok{.2}\ErrorTok{E}\OperatorTok{\textasciitilde{}}\NormalTok{P}\SpecialCharTok{\}}\SpecialStringTok{ per year. Dividing the total annual energy need of the electric vehicle fleet by the energy available from each turbine yields the number of turbines that will be needed: }\SpecialCharTok{\{}\NormalTok{n\_turbines}\SpecialCharTok{:}\FloatTok{.2}\ErrorTok{E}\OperatorTok{\textasciitilde{}}\NormalTok{P}\SpecialCharTok{\}}\SpecialStringTok{.}
\SpecialStringTok{"""}\NormalTok{))}
\end{Highlighting}
\end{Shaded}

Accounting for the plug-to-wheel efficiency of an electric vehicle, the
total energy needed to run the fleet will be 1.35E+09 MW·h per year, or
4.87E+12 MJ.

After accounting for the system efficiency of each turbine, the energy
that will be availabe from each will be 1.33E+01 MJ per year. Dividing
the total annual energy need of the electric vehicle fleet by the energy
available from each turbine yields the number of turbines that will be
needed: 3.66E+11.

\hypertarget{chapter-7}{%
\section{Chapter 7}\label{chapter-7}}

\hypertarget{section-7}{%
\subsection{7.1.}\label{section-7}}

\textbf{\emph{Explain the difference between primary energy and end-use
energy.}}

Primary energy is the energy released by breaking chemical bonds in raw
fuels, or the solar/wind/geothermal that is naturally available as
renewable energy. End-use energy is the energy consumed by various
applications after primary energy has been converted, transmitted, and
distribution losses have been accounted for.

\hypertarget{section-8}{%
\subsection{7.4.}\label{section-8}}

\textbf{\emph{Identify four ways that electrifying all energy and
providing the electricity with 100 percent wind, water, and solar
reduces end-use power demand. }}

\begin{enumerate}
\def\labelenumi{\arabic{enumi}.}
\tightlist
\item
  Electric vehicles are more efficient than combustion vehicles, so they
  need less energy to run.
\item
  In producing high temperature heat in industiral settings, electric
  appliances are also far more efficient than current combustion-based
  techniques, which decreases the end use demand.
\item
  End use demand is also decreased by using heat-pumps, which rely on
  heat transfer to create comforable infoor temperatures, rather than
  generating heat from fuel.
\item
  Using renewable sources eliminates the need for energy to mine,
  transport and process fossil fuels, which decreases end-use demand
\end{enumerate}

\hypertarget{section-9}{%
\subsection{7.10.}\label{section-9}}

\textbf{\emph{If a wind farm produces \$50,000/y in annual revenue for
30 years, what is the present value of that income, assuming a private
discount rate of 4 percent? }}

\begin{Shaded}
\begin{Highlighting}[]
\CommentTok{\# present\_val = future\_val * (1/(1+r)\^{}n)}
\NormalTok{future\_val }\OperatorTok{=} \DecValTok{50\_000}
\NormalTok{n }\OperatorTok{=} \DecValTok{30} \CommentTok{\# number of periods }
\NormalTok{r }\OperatorTok{=} \FloatTok{0.04} \CommentTok{\# discount rate }
\NormalTok{present\_val }\OperatorTok{=}\NormalTok{ future\_val }\OperatorTok{*}\NormalTok{ (}\DecValTok{1}\OperatorTok{/}\NormalTok{(}\DecValTok{1}\OperatorTok{+}\NormalTok{r)}\OperatorTok{**}\NormalTok{n)}
\NormalTok{display(Markdown(}\SpecialStringTok{f"""}
\SpecialStringTok{The present value of the income is  $}\SpecialCharTok{\{}\NormalTok{present\_val}\SpecialCharTok{:.0f\}}\SpecialStringTok{.}
\SpecialStringTok{"""}\NormalTok{))}
\end{Highlighting}
\end{Shaded}

The present value of the income is \$15416.

\hypertarget{section-10}{%
\subsection{7.11.}\label{section-10}}

\textbf{\emph{Estimate the cost per unit energy of a Pr = 5 MW nameplate
capacity wind turbine with a capital cost of \$1,200/kW, a capacity
factor of CF = 34 percent, an O\&M cost of \$30/kW-y, and a lifetime of
30 years. Assume the discount rate is 3.5 percent, the construction time
is 2 years, and the decommissioning cost is \$13/kW. Also assume
transmission, distribution, downtime, and array losses are 10 percent.
Hint: Construct a table like Table 7.9. }}

\begin{Shaded}
\begin{Highlighting}[]
\NormalTok{Pr }\OperatorTok{=} \DecValTok{5000} \CommentTok{\# kw {-}\textgreater{} nameplate capacity }
\NormalTok{P\_capital }\OperatorTok{=} \DecValTok{1200} \CommentTok{\# $/kw {-}\textgreater{} capital cost }
\NormalTok{CF }\OperatorTok{=} \DecValTok{30}\OperatorTok{/}\DecValTok{100} \CommentTok{\# percent {-}\textgreater{} capacity factor }
\NormalTok{OandM }\OperatorTok{=} \DecValTok{10} \CommentTok{\# $/kW/year {-}\textgreater{} O+M cost }
\NormalTok{i }\OperatorTok{=} \DecValTok{3}\OperatorTok{/}\DecValTok{100} \CommentTok{\# percent {-}\textgreater{}  private discount rate}
\NormalTok{T\_construct }\OperatorTok{=} \DecValTok{1} \CommentTok{\# year {-}\textgreater{} construction time }
\NormalTok{F }\OperatorTok{=} \DecValTok{10} \CommentTok{\# $/kW {-}\textgreater{} decomissioning cost }
\NormalTok{losses }\OperatorTok{=} \DecValTok{10}\OperatorTok{/}\DecValTok{100} \CommentTok{\# percent {-}\textgreater{} transmission, dist, and downtime losses }
\NormalTok{T\_life }\OperatorTok{=} \DecValTok{30} \CommentTok{\# project lifetime }

\NormalTok{T\_life\_arr }\OperatorTok{=}\NormalTok{ np.arange(}\DecValTok{1}\NormalTok{, T\_life}\OperatorTok{+}\DecValTok{1}\NormalTok{)}
\NormalTok{i\_factor }\OperatorTok{=}\NormalTok{ (}\DecValTok{1} \OperatorTok{+}\NormalTok{ i )}\OperatorTok{**}\NormalTok{(T\_life\_arr)}
\NormalTok{CRF }\OperatorTok{=}\NormalTok{( i }\OperatorTok{*}\NormalTok{ i\_factor ) }\OperatorTok{/}\NormalTok{ (i\_factor  }\OperatorTok{{-}} \DecValTok{1}\NormalTok{ )}

\CommentTok{\# print(CRF, T\_life\_arr, len(CRF))}

\CommentTok{\# energy available per year }
\NormalTok{Et }\OperatorTok{=}\NormalTok{ Pr }\OperatorTok{*}\NormalTok{ CF }\OperatorTok{*}\NormalTok{ st\_year\_hours }\OperatorTok{*}\NormalTok{ (}\DecValTok{1} \OperatorTok{{-}}\NormalTok{ losses)}

\CommentTok{\# presnet value of energy }
\NormalTok{Et\_present }\OperatorTok{=}\NormalTok{ Et}\OperatorTok{/}\NormalTok{i\_factor}

\CommentTok{\# costs}
\NormalTok{A\_capital }\OperatorTok{=}\NormalTok{ P\_capital }\OperatorTok{*}\NormalTok{ CRF}
\NormalTok{A\_decom }\OperatorTok{=}\NormalTok{ (F}\OperatorTok{/}\NormalTok{i\_factor)}\OperatorTok{*}\NormalTok{ CRF}
\NormalTok{OM\_cost }\OperatorTok{=}\NormalTok{ [}\DecValTok{0}\NormalTok{] }\OperatorTok{+}\NormalTok{ [OandM }\OperatorTok{*}\NormalTok{ Pr]}\OperatorTok{*}\NormalTok{(T\_life }\OperatorTok{{-}} \DecValTok{1}\NormalTok{)}

\NormalTok{costs }\OperatorTok{=}\NormalTok{ A\_capital }\OperatorTok{+}\NormalTok{ A\_decom }\OperatorTok{+}\NormalTok{ OM\_cost}

\NormalTok{LCOE }\OperatorTok{=}\NormalTok{ (np.}\BuiltInTok{sum}\NormalTok{(costs}\OperatorTok{/}\NormalTok{i\_factor))}\OperatorTok{/}\NormalTok{(np.}\BuiltInTok{sum}\NormalTok{(Et\_present}\OperatorTok{/}\NormalTok{i\_factor))}

\NormalTok{display(Markdown(}\SpecialStringTok{f"""}
\SpecialStringTok{The levelized cost of energy is $}\SpecialCharTok{\{}\NormalTok{LCOE}\SpecialCharTok{:.3f\}}\SpecialStringTok{.}
\SpecialStringTok{"""}\NormalTok{))}

\CommentTok{\# print(CRF, Et, LCOE)}
\end{Highlighting}
\end{Shaded}

The levelized cost of energy is \$0.006.

\hypertarget{chapter-8}{%
\section{Chapter 8}\label{chapter-8}}

\hypertarget{section-11}{%
\subsection{8.4.}\label{section-11}}

\textbf{\emph{List four methods of changing the electric power system
when instantaneous electric power generation plus storage cannot meet
instantaneous demand during some times of the year.}}

\begin{enumerate}
\def\labelenumi{\arabic{enumi}.}
\tightlist
\item
  Over-generation of energy
\item
  Energy storage
\item
  A highly interconnected grid enabling long-distance transmission
\item
  Demand-response
\end{enumerate}

\hypertarget{section-12}{%
\subsection{8.6.}\label{section-12}}

\textbf{\emph{Why should bundling of wind, solar, geothermal, and
hydroelectric power significantly help to match power demand on the grid
with power supply?}}

These renewable sources of energy often produce output at complementary
times. For instance, the high pressure systems that occur when the sun
is shining are often absent when the wind is blowing. Having a system
where all sources are interconnected can smooth the supply curve, and
ensure that when exess power is produced it can be stored.

\hypertarget{section-13}{%
\subsection{8.7.}\label{section-13}}

\textbf{\emph{Calculate the number of heating degree days and cooling
degree days over a four-day period relative to a reference temperature
of 18 oC if the outdoor air temperatures on each day are --5ºC, 0ºC,
30ºC, and 35ºC, respectively. }}

\begin{Shaded}
\begin{Highlighting}[]
\NormalTok{temp1 }\OperatorTok{=} \OperatorTok{{-}}\DecValTok{5}
\NormalTok{temp2 }\OperatorTok{=} \DecValTok{0} 
\NormalTok{temp3 }\OperatorTok{=} \DecValTok{30} 
\NormalTok{temp4 }\OperatorTok{=} \DecValTok{35}

\NormalTok{ref }\OperatorTok{=} \DecValTok{18}
\NormalTok{hdds }\OperatorTok{=}\NormalTok{ (ref }\OperatorTok{{-}}\NormalTok{ temp1) }\OperatorTok{+}\NormalTok{ (ref }\OperatorTok{{-}}\NormalTok{ temp2)}

\NormalTok{cdds }\OperatorTok{=}\NormalTok{ (temp3 }\OperatorTok{{-}}\NormalTok{ ref) }\OperatorTok{+}\NormalTok{ (temp4 }\OperatorTok{{-}}\NormalTok{ ref)}

\NormalTok{display(Markdown(}\SpecialStringTok{f"""}
\SpecialStringTok{The number of heating degree days is }\SpecialCharTok{\{}\NormalTok{hdds}\SpecialCharTok{:.0f\}}\SpecialStringTok{ºC and the number of cooling degree days is }\SpecialCharTok{\{}\NormalTok{cdds}\SpecialCharTok{:.0f\}}\SpecialStringTok{ºC.}
\SpecialStringTok{"""}\NormalTok{))}
\end{Highlighting}
\end{Shaded}

The number of heating degree days is 41ºC and the number of cooling
degree days is 29ºC.

\hypertarget{section-14}{%
\subsection{8.10.}\label{section-14}}

\textbf{\emph{What WWS technologies can be used to perform regulation
services?}}

Hydropwer, batteries, and storage from pumped hydropower, flywheel,
compressed air, and gravitational storage with solid masses.

\hypertarget{chapter-9}{%
\section{Chapter 9}\label{chapter-9}}

\hypertarget{section-15}{%
\subsection{9.1.}\label{section-15}}

\textbf{\emph{List four policy measures that could be implemented to
encourage the expansion of WWS energy systems.}}

\begin{enumerate}
\def\labelenumi{\arabic{enumi}.}
\tightlist
\item
  Renewable portfolio standards requiring certain fraction of ppwer
  generation to come from clean energy sources
\item
  Financial uncentives and laws for increasing energy efficiency and
  reducing energy use
\item
  Laws requiring demand response
\item
  Feed-in tariffs that cover the difference between electricity
  gemeration cost and wholesale electricity prices
\end{enumerate}

\hypertarget{section-16}{%
\subsection{9.2.}\label{section-16}}

\textbf{\emph{List four barriers that could slow the large-scale
implementation of clean, renewable energy.}}

\begin{enumerate}
\def\labelenumi{\arabic{enumi}.}
\tightlist
\item
  Long term vested interests from current energy infrastructure
\item
  Zoning issues
\item
  Countries engaged in conflict
\item
  Countireis with substantial poverty
\end{enumerate}

\hypertarget{section-17}{%
\subsection{9.3.}\label{section-17}}

\textbf{\emph{What fossil-fuel-based technologies will likely take the
longest to transition to electricity or electrolytic-hydrogen
alternatives?}}

Long haul airfraft will take the longest, estimated between between 2030
and 2040, to transition as fuel cell sizes and efficiencis will need to
improve

\hypertarget{section-18}{%
\subsection{9.4.}\label{section-18}}

\textbf{\emph{If the cumulative anthropogenic carbon emissions at the
end of 2018 were 2,155 Gt-CO2, and the limit to avoid 1.5 oC global
warming is 2,400 Gt-CO2, by what year will global warming reach 1.5 oC
if the anthropogenic emission rate from fossil-fuel combustion, cement
manufacturing, and land use change is 40 Gt-CO2/y?}}

\begin{Shaded}
\begin{Highlighting}[]
\NormalTok{co2\_2018 }\OperatorTok{=} \DecValTok{2\_155} \CommentTok{\# Gt{-}CO2}
\NormalTok{co2\_lim }\OperatorTok{=} \DecValTok{2\_400} \CommentTok{\# Gt{-}CO2 1.5ºC limit }
\NormalTok{em\_rate }\OperatorTok{=} \DecValTok{40} \CommentTok{\# Gt{-}CO2 emitted / year }

\NormalTok{co2\_remaining }\OperatorTok{=}\NormalTok{ co2\_lim }\OperatorTok{{-}}\NormalTok{ co2\_2018 }
\NormalTok{years\_remaining }\OperatorTok{=}\NormalTok{ co2\_remaining}\OperatorTok{/}\NormalTok{ em\_rate }
\NormalTok{year\_reached }\OperatorTok{=}\NormalTok{ years\_remaining }\OperatorTok{+} \DecValTok{2018}

\NormalTok{display(Markdown(}\SpecialStringTok{f"""}
\SpecialStringTok{The world will reach the 1.5º global warming limit about }\SpecialCharTok{\{}\NormalTok{years\_remaining}\SpecialCharTok{:.3f\}}\SpecialStringTok{ years from now, so in }\SpecialCharTok{\{}\NormalTok{year\_reached}\SpecialCharTok{:.0f\}}\SpecialStringTok{.}
\SpecialStringTok{"""}\NormalTok{))}
\end{Highlighting}
\end{Shaded}

The world will reach the 1.5º global warming limit about 6.125 years
from now, so in 2024.



\end{document}
